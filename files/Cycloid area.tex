\documentclass[11pt, oneside]{article} 
\usepackage{geometry}
\geometry{letterpaper} 
\usepackage{graphicx}
	
\usepackage{amssymb}
\usepackage{amsmath}
\usepackage{parskip}
\usepackage{color}
\usepackage{hyperref}

\graphicspath{{/Users/telliott/Github/figures/}}
% \begin{center} \includegraphics [scale=0.4] {gauss3.png} \end{center}

\title{Cycloid area}
\date{}

\begin{document}
\maketitle
\Large

%[my-super-duper-separator]

We imagine a bicycle with one tire marked at a particular point on the rim, say with fluorescent paint or a small light.  Time starts at $t = 0$ with that point $P$ in contact with the $x$ axis at $(0,0)$.  Then we start rolling the bike.  As the tire rotates our fixed point $P$ on the rim traces a curve called the cycloid.
\begin{center} \includegraphics [scale=0.6] {cycloid.png} \end{center}

We want to find equations that give the position of the point $P$ as a function of time.  
\begin{center} \includegraphics [scale=0.5] {cycloid2.png} \end{center}

It turns out that there is no simple expression for $y$ in terms of $x$.  However, there is a fairly simple \emph{parametrization} of the curve, yielding parametric equations $x(t)$, $y(t)$ (or $\theta$ as labeled in the diagram).

The unusual thing is that we will designate $t = 0$ as being the time when the point $P$ is at the bottom, and then the angle increasing in the clockwise direction, opposite to the usual case.  The circle has radius $a$, for the general case.

\begin{center} 
\includegraphics [scale=0.6] {cycloid2.png}
\includegraphics [scale=0.4] {cycloid3.png} 
\end{center}

It's a bit tricky to think about, but the equations for both $y$ and $x$ as a function of $t$ have two terms.  $y$ is easier.  The distance of point $P$ above the $x-$axis is the radius of the circle, \emph{minus} the length shown in the right panel as $a \cos t$.
\[ y = a - a \cos t \]

The distance of the center of the circle from the $y-$ axis is $at$, the distance along the arc of the circle from $P$ to the point currently on the bottom.  But we must account for the distance of $P$ left or right of the center, which is $- a \sin t$.  Thus:
\[ x = at - a \sin t \]

Now, going back to the original diagram, the area under the curve is just the integral of $y$.
\begin{center} \includegraphics [scale=0.6] {cycloid.png} \end{center}

That is
\[ A = \int y \ dx \]
But we change variables to $t$.  What is $dx$?  $dx = (a - a \cos t) \ dt$.
\[ A = \int (a - a \cos t) (a - a \cos t) \ dt \]
\[ A = a^2 \int (1 - \cos t)^2 \ dt \]

The bounds on the integral are given by the fact that we want one complete revolution of the circle.
\[ A = a^2 \int_0^{2 \pi} (1 - \cos t)^2 \ dt \]

That will give a nice simplification, since any simple trigonometric function gives an integral of zero over that range.  Let's see

\[ (1 - \cos t)^2 = 1 - 2 \cos t + \cos^2 t \]

We will need antiderivatives.  The first two are easy, but what about $\cos^2$?

Recall the product rule:  $(uv)' = u'v + uv'$.  If you play around with various products, you will encounter:
\[ \frac{d}{dx} \sin x \cos x = \cos^2 x - \sin^2 x = 2 \cos^2 x - 1 \]
From this we can rearrange and integrate to obtain:
\[ 2 \int \cos^2 x \ dx = x + \sin x \cos x \]

Putting it all together we have:
\[ A = a^2 \int_0^{2 \pi} (1 - \cos t)^2 \ dt \]
\[ = a^2  \ [ \ t - 2 \sin t + \frac{1}{2}(t + \sin t \cos t) \bigg |_0^{2 \pi} \ ] \]

Those terms with $\sin$ and $\cos$ are just zero over the period $2 \pi$.  That leaves
\[ A = a^2 \ \frac{3}{2} t \bigg |_0^{2 \pi} = 3 \pi a^2 \]

A simple, beautiful answer.
\begin{center} \includegraphics [scale=0.25] {cycloid4.png} \end{center}
In this figure, the radius is $R$, so the circle is $\pi R^2$.  We pick up two copies of $(1/2) \pi R \cdot 2R$ which gives us a total of $3 \pi R^2$.

\end{document}