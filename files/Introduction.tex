\documentclass[11pt, oneside]{article} 
\usepackage{geometry}
\geometry{letterpaper} 
\usepackage{graphicx}
	
\usepackage{amssymb}
\usepackage{amsmath}
\usepackage{parskip}
\usepackage{color}
\usepackage{hyperref}

\graphicspath{{/Users/telliott/Github/figures/}}
% \begin{center} \includegraphics [scale=0.4] {gauss3.png} \end{center}

\title{Introduction}
\date{}

\begin{document}
\maketitle
\Large

%[my-super-duper-separator]
This was supposed to be a short book, an exploration of problems like the volume of the cone and sphere, or even just the area of a circle, with some simple physics thrown in.  These questions contain within them the heart of calculus:  infinities both large and small.  I imagine myself looking over Archimedes' shoulder as he explains it to me.

I wrote many of the early chapters originally as short explanations for my son Sean as he studied calculus in high school.  It bothers me that so often the good stuff gets left out --- the ideas which make you go ... wow.  Now, years later, I still find a lot of pleasure in trying to understand what Kepler and Newton did.  It took a genius to figure it out the first time, but it is within anyone's grasp to appreciate what they found.

Then I thought, why not include other favorite problems like the area of the ellipse, the "headlight" problem for the parabola, or the reflective property of the ellipse, and the length and area under the cycloid curve (the "light on a bicycle wheel"). These are problems where calculus easily produces answers that can be checked by more elaborate geometric arguments.  In fact, this book might as well be titled .. \emph{Best of Calculus and Geometry}.

So here we are, with a somewhat longer book.

In the introduction to his book \emph{Calculus}, Morris Kline says
\begin{quote}Anyone who adds to the plethora of introductory calculus texts owes an explanation, if not an apology, to the mathematical community.\end{quote}

I think of this book as akin to ultralight backpacking.  We shed weight so as to ascend peaks rapidly, skimming the best of calculus --- focusing on geometry and physics, and slinging differentials with abandon.  Epsilon is a bit player in the production.  Starting with an intuitive notion of adding up many small pieces, we put integrals to work early solving problems.

Going fast allows time to get a view of sophisticated topics, among others, line integrals for work and flux, Newton's proof that a spherical mass acts as a point mass, and integration of a parametrized surface like the torus.  Not to mention Kepler's Laws, and a derivation of the Gaussian distribution from first principles.

We do not disdain proof.  Proof is central to the enterprise.  We prove the Pythagorean Theorem, and the quotient rule for derivatives, as well as Green's Theorem.  There is a fun chapter on induction.  We prove that $\pi$ is a constant.  In fact, the word "proof" appears nearly 200 times in the text and one of its most interesting features is the natural use of proofs that I have tried to make as simple and easy to follow as possible.

My favorite authors on calculus are Morris Kline, Richard Hamming, and Gil Strang.  Sylvanus Thompson's simple book is my favorite first text, and it's even a Project Gutenberg project:

\url{https://www.gutenberg.org/files/33283/33283-pdf.pdf}

Having said what I like, briefly, here are some things I don't like.

The rigorous approach to calculus pioneered by Cauchy in the 1820's and exported to American schools by Richard Courant in the 1940's is a bad idea.  We must motivate rigorous proof by demonstrating utility first.  As Ian Stewart says, "proofs come \emph{after} understanding."  Courant's method is the way to teach the subject the second or even third time through.

Thompson:

\begin{quote}
You don't forbid the use of a watch to every person who does not know how to make one. You don't object to the musician playing on a violin that he has not himself constructed. You don't teach the rules of syntax to children until they have already become fluent in the use of speech. It would be equally absurd to require general rigid demonstrations to be expounded to beginners in the calculus.\end{quote}

A second thing I dislike is calculus problems that are gratuitously arithmetic.  Calculus consists of bright ideas, not complicated ones;  if the computation is difficult, it's usually \emph{not} a good problem.  Also, a good problem often is one with a physical or practical foundation.  Having said that, if a course could integrate elementary programming with calculus, I would be very happy.

Finally, a saying attributed to Manaechmus (speaking to Alexander the Great), "there is no royal road to geometry".  Which means, practically, learning mathematics requires that you follow the argument with pencil and paper and work out each step yourself, to your own satisfaction.  That is the only way of really learning, and at heart, one of the reasons I wrote this book.

I express my sincere thanks to the authors of my favorite books, which are listed in the references and mentioned at various places in the text.  Almost everything in here was appropriated from them, and styled to my taste.  I offer my profound thanks also to Eugene Colosimo, S.J.  He was, for me, the best of a bunch of very special teachers.

If I stole your figure off the internet, I'm sorry.  I intended to redraw it but have not yet found the time.

Update (Feb 2020):  I have broken the original book into parts.  

Part one is Geometry and topics for Precalculus.  It is here:

\url{https://github.com/telliott99/precalculus}

Part two is single variable Calculus.  It is here:

\url{https://github.com/telliott99/calculus}

The original book is here:

\url{https://github.com/telliott99/calculus_book}

\end{document}