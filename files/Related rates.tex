\documentclass[11pt, oneside]{article} 
\usepackage{geometry}
\geometry{letterpaper} 
\usepackage{graphicx}
	
\usepackage{amssymb}
\usepackage{amsmath}
\usepackage{parskip}
\usepackage{color}
\usepackage{hyperref}

\graphicspath{{/Users/telliott/Github/figures/}}
% \begin{center} \includegraphics [scale=0.4] {gauss3.png} \end{center}

%break
\title{Related rates}
\date{}

\begin{document}
\maketitle
\Large

%[my-super-duper-separator]

One simple form of related rates problem has two objects moving at right angles from each other, with positions and speeds given in terms of the origin. 

 For example:
"$A$ moves west at $x$ miles per hr, his current position is $x_0$ miles west of the origin, while $B$ moves south at $y$ miles per hr, and his current position is $y_0$ miles south of the origin.  At what rate are they moving apart?"

For the distance, we use Pythagoras:

\[ h^2 = x^2 + y^2 \]
All three values are functions of time so
\[ 2h h' = 2 x x' + 2 y y' \]
\[ h' = \frac{1}{h} (x x' + y y') \]
We will have to calculate $h$ from $x_0$ and $y_0$.

Another simple related rates problems involves two quantities with an equation relating the two quantities, e.g. the volume and radius of a sphere, where the sphere is a "balloon being inflated" or something

\[ V = \frac{4}{3} \pi r^3 \]
\[ \frac{dV}{dt} = 4 \pi r^2 \ \frac{dr}{dt} \]
or as usually stated in these problems
\[ V' = 4 \pi r^2 \ r' \]
If we know $V'$ and $r$ we can calculate $r'$.  Usually, rather than give you $r$ they will give you $V$, so then
\[ r = (\frac{4}{3} V)^{1/3} \]

A related problem :) is where the object is a cone (maybe inverted) and it's filling up with a fluid.  

\begin{center} \includegraphics [scale=0.5] {cone_rates.png} \end{center}

Here, the formula for the volume of a cone is

\[ V = \frac{1}{3} \pi r^2 h \]

The problem is that since $r$ and $h$ depend on each other, we can't simply do 
\[ V' = \frac{1}{3} \pi r^2 h' \]
(this is wrong!)

In this case it's important to realize that the radius $r$ and the height $h$ of the fluid at its current level have the same ratio as the radius $R$ and height $H$ of the container.

\[ \frac{r}{h} = \frac{R}{H} \]
\[ r = \frac{R}{H} h \]
\[ h = \frac{H}{R} r \]

so we can substitute using the relationship between $r$ and $h$

\[ V = \frac{1}{3} \pi r^2 \frac{H}{R} r = \frac{1}{3} \pi \frac{H}{R} r^3 \]

Alternatively, we can eliminate $r$

\[ V = \frac{1}{3} \pi \frac{R^2}{H^2} h^3  \]

For example, with the figure above, ($R=5$ and $H=14$ feet), and given water is draining from the tank at $V'=-2 ft^3$ per hour

"At what rate is the depth of the water in the tank changing when the depth of the water is 6 ft?"

\[ V = \frac{1}{3} \pi \frac{R^2}{H^2} h^3  \]
\[ V' = \pi \frac{R^2}{H^2} h^2 h' \]

We're given $V'$ and $h$, $H$ and $R$, so can solve for $h'$.

The second question is "At what rate is the radius of the top of the water in the tank changing when the depth of the water is 6 ft?"

We need $r'$ given $h$ (and $R$, $H$, and $V'$)

\[ V = \frac{1}{3} \pi \frac{H}{R} r^3 \]
\[ V' = \pi \frac{H}{R} r^2 r' \]

\[ r = \frac{R}{H} h \]
\[ r^2 = (\frac{R}{H})^2 h^2 \]

Plugging in
\[ V' = \pi \frac{R}{H} h^2 r' \]
\vspace{10mm}

Here is another related rates problem.  An airplane and a parachutist are at the same height currently, and in the same direction as you look at them.  

\begin{center} \includegraphics [scale=0.5] {rr1.png} \end{center}

The airplane moves away from you at $500$ ft/s.  The parachutist is floating downward at $-10$ ft/s and will land $1000$ ft away from you.  The current value of $h = 2000$ ft.  The current value of $p = 8000$ ft.  Find $d \theta/dt$.

\[ p(t) = p_0 + 500 t \]
\[ h(t) = h_0 - 10 t  \]
\[ p' = 500 \]
\[ h' = -10 \]

Find equations for the angles involved
\[ \tan s = \frac{2000}{p} \]
we use the constant value of $2000$ rather than $h$, which will vary.
\[ u = s + \theta \]
\[ \tan u = \frac{h}{1000} \]
Take the derivatives.  For the airplane
\[ \tan s = \frac{2000}{p} \]
\[ \frac{d}{dt} \tan s = \sec^2 s \frac{ds}{dt} = \frac{d}{dt} \frac{2000}{p}  = -2000 \frac{1}{p^2} \frac{dp}{dt}  \]
\[ \frac{ds}{dt} = -2000 \frac{1}{p^2} \frac{dp}{dt} \cos^2 s \]

In the above equation, we know $p=8000$ and $dp/dt = 500$.  We have to find the cosine.  If $\tan s=1/4$ then $\cos s = \sqrt{16/17}$.  
\[ \frac{ds}{dt} = -2000 \ \frac{1}{8000^2} \ 500 \ \frac{16}{17} \]
\[ \frac{ds}{dt} = -\frac{1}{4} \frac{1}{16} \frac{16}{17} = -\frac{1}{68} = - 0.0147\]

For the parachutist

\[ \frac{d}{dt} \tan u = \sec^2 u \frac{du}{dt} = \frac{d}{dt} \frac{h}{1000}  = \frac{1}{1000} \frac{dh}{dt}  \]
\[ \frac{du}{dt} = \frac{1}{1000} \frac{dh}{dt}  cos^2 u \]

In the above equation, we know $dh/dt = -10$.  We have to find the cosine.  If $\tan u=2$ then $\cos u = 1/\sqrt{5}$.  So
\[ \frac{du}{dt} = 0.001 \  (-10) \ \frac{1}{5}  = - 0.002 \]

Since $\theta = u - s$
\[ \frac{d \theta}{dt} = \frac{du}{dt} -  \frac{ds}{dt} = - 0.002 + 0.0147 =  0.0127 \]
The angle $\theta$ between plane and parachutist is \emph{increasing} with time (about 3/4 of a degree per second).

\end{document}